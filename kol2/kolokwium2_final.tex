\documentclass[12pt]{article}

\usepackage{geometry}
\usepackage{amsmath}
\usepackage{amssymb}
\usepackage{enumitem}
\usepackage{fancyhdr}
\usepackage[utf8x]{inputenc}
\usepackage[T1]{fontenc}
\usepackage[polish]{babel}
\usepackage{multicol}

\renewcommand{\headrulewidth}{1.5pt}
\pagestyle{fancy}

\fancyhead[LE,RO]{Share\LaTeX}
\fancyhead[RE,LO]{Guides and tutorials}
\fancyfoot[CE,CO]{\leftmark}
\fancyfoot[LE,RO]{\thepage}
\setlength{\parindent}{0pt}

\lhead{\textsc{Programowanie}}
%\chead{}
\rhead{\textsc{Kolokwium 2, 25.05.2017}}

\newcounter{zadanie}
\newcommand{\zadanie}{\stepcounter{zadanie}\paragraph*{Zadanie \arabic{zadanie}. (10 pkt)}}

\begin{document}

\small Kod źródłowy rozwiązania każdego z zadań zapisz z oddzielnym pliku *.cpp o nazwie \textsc{indeks}\_\textsc{nr zadania}. Programy nie kompilujące się otrzymują {\bf zero} punktów. Można korzystać z Internetu, jednak zabronione są wszelkie formy kontaktowania się. Powodzenia!

\zadanie
Napisz klasę \textsf{Naukowiec} zawierającą nazwisko, liczbę publikacji oraz cytowań naukowca. Klasa powinna zapewniać następujące operacje:
\begin{itemize}
\item konstruktor trójargumentowy inicjalizujący nazwisko, liczbę publikacji oraz cytowań,
\item funkcję składową \textsf{indeks} zwracającą średnią liczbę cytowań na jedną publikację,
\item funkcję składową \textsf{publikuj} zwiększającą liczbę publikacji o zadaną wartość,
\item operator preinkrementacji \textsf{++} zwiększający o 1 liczbę cytowań naukowca,
\item operator \textsf{<} porównujący naukowców. Lepszy jest naukowiec z większą liczbą cytowań, a jeżeli są takie same, to ten dla którego funkcja \textsf{indeks} zwraca większą wartość.
\item operator \textsf{<<} wypisywania danych naukowca do strumienia typu \textsf{ostream} oraz operator \textsf{>>} wczytywania danych naukowca ze strumienia \textsf{istream}.
\end{itemize}
\textsc{Przykład użycia klasy:}
\begin{verbatim}
int main() {
   Naukowiec Bogdan("Bogdan",50,300), Czeslaw;
   cin >> Czeslaw;
   cout << Czeslaw.indeks();
   Bogdan.publikuj(3);
   if(Bogdan < ++Czeslaw)
      cout << Czeslaw << endl;
   else
      cout << Bogdan << endl;
}
\end{verbatim}

\zadanie
Sprawdzanie kolokwium z programowania to bardzo ciężka praca. Wyniki kolokwium znajdują się w pliku tekstowym, zapisanym w następującym formacie: w każdej linii znajduje się jedno imię i jedno nazwisko studenta, a następnie liczby punktów za kolejne zadania. Liczba zadań ani osób nie jest z góry znana, wiadomo jednak, że liczba zadań jest dla wszystkich studentów taka sama oraz plik z danymi nie jest pusty. Napisz program, który na podstawie tego pliku obliczy i wypisze do pliku \textsf{wyniki.txt} całkowitą liczbę punktów zdobytych przez każdego studenta oraz średnią liczbę punktów z każdego zadania. Nazwa pliku zawierającego dane wejściowe powinna być podawana jako argument wywołania programu. \textit{Wskazówka}: możesz użyć pojemnika \textsf{<vector>} z biblioteki STL oraz metod \textsf{clear} i \textsf{str} dla obiektu strumienia \textsf{istringstream}. \textsc{Przykład wejścia i wyjścia:} 
\begin{multicols}{2}
\textsc{Plik wejściowy:} (\textsf{dane.txt})
\begin{verbatim}
Jakub Wedrowycz 1.5 0 0.5
Semen Korczaszko 5.5 10 3.5
Jozef Paczenko 4 5 6.5

\end{verbatim}
\textsc{Plik wyjściowy:} (\textsf{wyniki.txt})
\begin{verbatim}
Jakub Wedrowycz 2
Semen Korczaszko 20
Jozef Paczenko 15.5
z1 4   z2 5   z3 3.5
\end{verbatim}
\end{multicols}

\zadanie
Napisz klasę \textsf{Statystyka} obliczająca średnią, odchylenie standardowe oraz medianę ciągu liczb o dowolnej długości. Klasa powinna udostępniać
funkcje:
\begin{itemize}
\item \textsf{void dodaj (double x)} — uwzględnia w obliczeniach nową liczbę \textsf{x},
\item \textsf{double srednia()} — przekazuje średnią arytmetyczną dotychczas zapamiętanych liczb,
\item \textsf{double sigma()} — przekazuje odchylenie standardowe dotychczas zapamiętanych liczb,
\item \textsf{double mediana()} — przekazuje wartość mediany dotychczas zapamiętanych liczb,
\item \textsf{void zeruj()} — usuwa zapamiętane liczby (przywraca początkowy zbiór pusty).
\end{itemize}

Napisz program, który losuje $N$ liczb rzeczywistych z rozkładu płaskiego w przedziale $[-k,k]$, gdzie $N,k\in\mathbb{N}$ są argumentami wywołania programu. Program powinien zwracać na wyjściu średnią, odchylenie standardowe oraz medianę wylosowanych liczb. \textit{Wskazówka}: możesz użyć pojemnika \textsf{<vector>} z biblioteki STL oraz funkcji \textsf{atoi} z biblioteki \textsf{<cstdlib>}. \textsc{Przykład użycia klasy:}
\begin{verbatim}
int main() {
   Statystyka stat;
   for (int i = 0; i<10; i++)
      stat.dodaj(i);
   cout << stat.srednia() <<' '<< stat.sigma() <<' '<< stat.mediana();
   stat.zeruj();
}
\end{verbatim}

\end{document}
