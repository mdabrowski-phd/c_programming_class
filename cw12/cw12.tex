\documentclass[12pt]{article}

\usepackage{geometry}
\usepackage{amsmath}
\usepackage{amssymb}
\usepackage{enumitem}
\usepackage{fancyhdr}
\usepackage[utf8x]{inputenc}
\usepackage[T1]{fontenc}
\usepackage[polish]{babel}
\usepackage{amsmath}
\usepackage{multicol}

\renewcommand{\headrulewidth}{1.5pt}
\pagestyle{fancy}

\fancyhead[LE,RO]{Share\LaTeX}
\fancyhead[RE,LO]{Guides and tutorials}
\fancyfoot[CE,CO]{\leftmark}
\fancyfoot[LE,RO]{\thepage}

\lhead{\textsc{Programowanie, grupa 1}}
%\chead{mdabrowski@fuw.edu.pl}
\rhead{\textsc{Ćwiczenia 12, 25.05.2017}}

\begin{document}

\small \textsc{Elementy języka C++:} pojemniki STL, powtórzenie klas oraz działań na plikach.

\begin{enumerate}

\item \textbf{Supermarket}\\
Napisz klasę \textsf{Koszyk} reprezentującą kasę sklepową i obliczającą koszt zakupów pojedynczego
klienta. Klasa powinna zapewniać następujące operacje:
\begin{itemize}
\item zerowanie przed rozpoczęciem obsługi kolejnego klienta — funkcja \textsf{void zeruj()}
\item dodanie nazwy i ceny towaru — funkcja \textsf{void dodaj(string nazwa, double wartosc)}
\item obliczenie łącznej należności za wszystkie zakupione towary — funkcja \textsf{double suma()}
\item wypisanie na standardowe wyjście listy zakupów — funkcja \textsf{void lista()}. Jeżeli
funkcją \textsf{dodaj} wprowadzono kilka takich samych przedmiotów, to w wydruku należy w pojedynczym
wierszu wypisać ich łączna wartość.
\end{itemize}
\textit{Wskazówka}: użyj pojemnika \textsf{<map>} z biblioteki STL.

\item \textbf{Histogram próby losowej}\\
Napisz klasę \textsf{Histogram} reprezentującą jednowymiarowy histogram o binach jednakowej szerokości.
Zaimplementuj:
\begin{itemize}
\item konstruktor pozwalający na zadanie przedziału histogramowania i liczby binów oraz zerujący liczby
zliczeń we wszystkich binach.
\item funkcję składową \textsf{reset} zerującą liczby zliczeń we wszystkich binach.
\item funkcję składową \textsf{insert} zwiększającą o jeden liczbę zliczeń w binie, w którym wypada liczba
rzeczywista zadana argumentem funkcji.
\item funkcję składową \textsf{count}, która wywołana bez argumentu zwraca całkowitą liczbę zliczeń w histogramie,
a wywołana z argumentem całkowitym zwraca liczbę zliczeń w binie o zadanym numerze.
\item funkcję składową \textsf{print} wypisującą liczby zliczeń we wszystkich binach do zadanego strumienia
wyjściowego. Dane kolejnych binów powinny znajdować się w kolejnych wierszach, a każdy wiersz
powinien zawierać numer binu oraz odpowiadającą mu liczbę zliczeń.
\end{itemize}
Następnie napisz program, który losuje 100 000 liczb rzeczywistych, z których każda jest sumą dwóch
liczb losowych z rozkładu płaskiego zawartego w przedziale od 0 do 1. Sporządź histogram tej próby losowej
dobierając przedział histogramowania oraz liczbę binów tak, aby jak najlepiej oddać kształt otrzymanego
rozkładu prawdopodobieństwa. \textit{Wskazówka}: uzyskany histogram wydrukuj na standardowe wyjście, a następnie
wykreśl go przy pomocy programu \textsf{gnuplot}, pisząc na przykład:
\textsf{plot ''histogram.txt'' with boxes}

\newpage

\item \textbf{Kodowanie symetryczne}\\
W plikach tekstowych każdy znak zapisany jest jako jednobajtowa liczba, zwana kodem znaku.
Jedną z metod szyfrowania wiadomości jest zastąpienie kodu każdego znaku jego bitową różnicą symetryczną
(\textsc{XOR}) z zadaną jednobajtową liczbą, zwaną kluczem. Aby odczytać wiadomość,
wystarczy ponownie wziąć bitową różnicę symetryczną zaszyfrowanych kodów z tym samym kluczem.
Zaszyfrowane wiadomości są zapisywane przy pomocy liczb oddzielonych spacjami. Szyfrowanie odbywa się znak po znaku, włączając spacje, tabulatory, znaki końca linii, itp.
Napisz program szyfrujący i rozszyfrowujący wiadomości tekstowe przy pomocy różnicy symetrycznej
z kluczem. Wiadomości, zarówno zaszyfrowane jak i niezaszyfrowane, zapisane są w plikach tekstowych. Program
powinien przyjmować cztery argumenty wywołania. Pierwszy to klucz, drugi to \textsf{e} lub \textsf{d} odpowiednio
dla szyfrowania i rozszyfrowywania, zaś trzeci i czwarty to nazwy plików wejściowego i wyjściowego.
\textit{Wskazówka}: operator \textsc{XOR} zapisujemy przy pomocy znaku \textsf{\^{}}.

\item \textbf{Dane pomiarowe}\\
Plik tekstowy z danymi ma następującą strukturę: każda liczba zapisana jest w formacie naukowym na 8 znakach, a pomiędzy liczbami znajdują się pojedyncze spacje. Plik nie zawiera pustych linii, jednak liczba kolumn ani wierszy nie jest z góry znana. Przyjmujemy jednak, że w pliku znajduje się przynajmniej jedna liczba. Napisz program, który na końcu każdego wiersza dopisze sumę wszystkich liczb z tego wiersza, a na końcu każdej kolumny dopisze sumę wszystkich liczb z tej kolumny. Oprócz tego na przecięciu dopisanych wiersza i kolumny powinna znaleźć się suma wszystkich liczb z oryginalnego pliku. Program powinien czytać dane ze standardowego wejścia, a wynik wypisywać na standardowe wyjście.
\begin{multicols}{2}
\textsc{Plik wejściowy:}
\begin{verbatim}
-1.0e+00  2.0e+01
 3.0e+01 -4.0e+00

\end{verbatim}
\textsc{Plik wyjściowy:}
\begin{verbatim}
-1.0e+00  2.0e+01  1.9e+01
 3.0e+01 -4.0e+00  2.6e+01
 2.9e+01  1.6e+00  4.5e+01
\end{verbatim}
\end{multicols}

\end{enumerate}

\vspace{0.3cm}
\small \textsc{Zadań domowych wyjątkowo brak.} Kolokwium dziś o godz. 17:00. Powodzenia!
\end{document}