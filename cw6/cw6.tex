\documentclass[12pt]{article}

\usepackage{geometry}
\usepackage{amsmath}
\usepackage{amssymb}
\usepackage{enumitem}
\usepackage{fancyhdr}
\usepackage[utf8x]{inputenc}
\usepackage[T1]{fontenc}
\usepackage[polish]{babel}
\usepackage{fancyvrb}

\renewcommand{\headrulewidth}{1.5pt}
\pagestyle{fancy}

\fancyhead[LE,RO]{Share\LaTeX}
\fancyhead[RE,LO]{Guides and tutorials}
\fancyfoot[CE,CO]{\leftmark}
\fancyfoot[LE,RO]{\thepage}

\lhead{\textsc{Programowanie, grupa 1}}
%\chead{mdabrowski@fuw.edu.pl}
\rhead{\textsc{Ćwiczenia 6, 6.04.2017}}

\begin{document}

\small \textsc{Elementy języka C++:} debuggowanie kodu programu używając CodeBlocks.

\begin{enumerate}

\item \textbf{Ciąg Fibonacciego 2 -- znajdź i napraw błąd}\\

\begin{Verbatim}[fontsize=\tiny]
#include <iostream>
using namespace std;

int fibonacci (int n){
	if ( n == 0 )
		return 1;
	return fibonacci( n - 1 ) + fibonacci( n - 2 );
}

int main(){
	int n;
	cout << "Enter the number to compute fibonacci for: " << endl;
	cin >> n;
	cout << fibonacci( n );
}
\end{Verbatim}

\item \textbf{Złe potęgowanie -- znajdź i napraw błąd}\\

\begin{Verbatim}[fontsize=\tiny]
#include <iostream>
using namespace std;

int exponent (int base, int exp){
	int running_value;
	for ( int i = 0; i < exp; i++ )
		running_value *= base;
	return base;
}

int main(){
	int base;
	int exp;
	cout << "Enter a base value: ";
	cin >> base;
	cout << "Enter an exponent: ";
	exponent( exp, base );
}
\end{Verbatim}

\item \textbf{Niekończąca się silnia  -- znajdź i napraw błąd}\\

\begin{Verbatim}[fontsize=\tiny]
#include <iostream>
using namespace std;

int main (){
	int factorial = 1;
	for ( int i = 0; i < 10; i++ )
		factorial *= i;
	int sum = 0;
	for ( int i = 0; i < 10; i++ )
		sum += i;
	int factorial_without_two = 1;
	for ( int i = 0; i < 10; i++ ){
		if ( i == 2 )
			continue;
		factorial_without_two *= i;
	}
	int sum_without_two = 0;
	for ( int i = 0; i < 10; i++ ){
		if ( i = 2 )
			continue;
		sum_without_two += i;
	}
} 
\end{Verbatim}

\item \textbf{Sumowanie tablicy  -- znajdź i napraw błąd}\\

\begin{Verbatim}[fontsize=\tiny]
#include <iostream>
using namespace std;

int sumValues (int values[], int n){
	int sum;
	for ( int i = 0; i <= n; i++ )
		sum += values[ i ];
	return sum;
}

int main(){
	int size;
	cout << "Enter a size: ";
	cin >> size;
	int values[size];
	int i;
	while ( i < size ){
		cout << "Enter value to add: ";
		cin >> values[ ++i ];
	}
	cout << "Total sum is: " << sumValues( values, size );
} 
\end{Verbatim}

\item \textbf{Korzystna lokata?  -- znajdź i napraw błąd}\\

\begin{Verbatim}[fontsize=\tiny]
#include <iostream>
using namespace std;

double computeInterest (double base_val, double rate, int years){
	double final_multiplier;
	for ( int i = 0; i < years; i++ ){
		final_multiplier *= (1 + rate);
	}
	return base_val * final_multiplier;
}

int main (){
	double base_val;
	double rate;
	int years;
	cout << "Enter a base value: ";
	cin >> base_val;
	cout << "Enter an interest rate: ";
	cin >> rate;
	cout << "Enter the number of years to compound: ";
	cin >> years;
	cout << "After " << years << " you will have " << 
	computeInterest( base_val, rate, years ) << " money" << endl;
} 
\end{Verbatim}

\end{enumerate}

\vspace{0.3cm}
\small \textsc{Zadań domowych wyjątkowo brak.} Kolokwium dziś o godz. 17:00. Powodzenia!

\end{document}