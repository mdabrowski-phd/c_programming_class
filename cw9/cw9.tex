\documentclass[12pt]{article}

\usepackage{geometry}
\usepackage{amsmath}
\usepackage{amssymb}
\usepackage{enumitem}
\usepackage{fancyhdr}
\usepackage[utf8x]{inputenc}
\usepackage[T1]{fontenc}
\usepackage[polish]{babel}

\renewcommand{\headrulewidth}{1.5pt}
\pagestyle{fancy}

\fancyhead[LE,RO]{Share\LaTeX}
\fancyhead[RE,LO]{Guides and tutorials}
\fancyfoot[CE,CO]{\leftmark}
\fancyfoot[LE,RO]{\thepage}

\lhead{\textsc{Programowanie, grupa 1}}
%\chead{mdabrowski@fuw.edu.pl}
\rhead{\textsc{Ćwiczenia 9, 4.05.2017}}

\begin{document}

\small \textsc{Elementy języka C++:} dynamiczne struktury danych -- tablica, lista powiązana.

\begin{enumerate}

\item \textbf{Obsługa łańcuchów o zmiennej długości}\\
Stwórz implementację opartą na dynamicznie alokowanych tablicach dla trzech funkcji obsługujących łańcuchy:

\textsf{append} -- funkcja wymaga podania łańcucha i znaku, a w wyniku swojego działania dołącza ten znak do końca łańcucha.

\textsf{concatenate} -- funkcja używa dwóch łańcuchów i dołącza znaki z drugiego do pierwszego.

\textsf{characterAt} -- funkcja wymaga użycia łańcucha oraz odpowiedniej liczby, zwracając znak znajdujący się na wskazywanej przez nią pozycji w łańcuchu (pierwszy znak łańcucha ma indeks równy zeru).

Napisz kod zakładając, że funkcja \textsf{characterAt} będzie wywoływana często, a dwie pozostałe stosunkowo rzadko. Względna wydajność operacji powinna odzwierciedlać częstotliwość ich stosowania.

\item \textbf{Obsługa rejestru studentów o nieznanej liczbie elementów}\\
Napisz funkcje przechowujące i modyfikujące kolekcję rejestrów studentów. Pojedynczy rejestr zawiera numer studenta i jego ocenę -- dane te są liczbami całkowitymi. Powinny zostać zaimplementowane następujące funkcje:

\textsf{addRecord} -- funkcja używa wskaźnika do kolekcji rejestrów studentów, zawierających numery studentów i ich oceny, aby dodać do niej nowy rejestr z wypełnionymi danymi.

\textsf{averageRecord} -- funkcja wymaga podania wskaźnika do kolekcji rejestrów studentów i zwraca średnią ocen studentów w kolekcji jako liczbę typu \textit{double}.

Kolekcja może mieć dowolny rozmiar. Operacja \textsf{addRecord} będzie wykonywana często, więc powinna zostać zaimplementowana w wydajny sposób.

\item \textbf{Implementacja łańcucha znaków przy użyciu listy powiązanej *}\\
Stwórz implementację łańcuchów tekstowych, która zamiast wykorzystywać dynamicznie alokowane tablice, używa listy powiązanej zawierającej znaki. Tak więc będziesz mieć listę powiązaną, w której właściwymi danymi będą pojedyncze znaki. Pozwoli to na operacje powiększania bez potrzeby ponownego tworzenia łańcucha od nowa. Rozpocznij od zaimplementowania funkcji \textsf{append} i \textsf{characterAt}.


\newpage
\rhead{\textsc{Zadania domowe 9}}

\item \textbf{Obsługa łańcuchów o zmiennej długości 2}\\
Stwórz funkcję \textsf{substring} z następującymi trzema parametrami: zmienną typu \textsf{arrayString}, liczbą całkowitą oznaczającą początkową pozycję oraz liczbą całkowitą określającą długość łańcucha. Funkcja powinna zwrócić wskaźnik do nowo przydzielonego bloku pamięci zawierającego tablicę znaków. Musi ona zawierać znaki pochodzące z oryginalnego łańcucha, poczynając od określonej pozycji i o podanej długości. Pierwotny łańcuch nie powinien zostać zmodyfikowany.

\item \textbf{Obsługa łańcuchów o zmiennej długości 3}\\
Stwórz funkcję \textsf{replaceString}, która wykorzystuje trzy parametry, każdy o typie \textsf{arrayString}: \textsf{source}, \textsf{target} i \textsf{replaceText}. Funkcja zamienia każde wystąpienie łańcucha \textsf{target} w łańcuchu \textsf{source} łańcuchem \textsf{replaceText}.

\item \textbf{Obsługa rejestru studentów o nieznanej liczbie elementów 2}\\
Napisz funkcję \textsf{removeRecord}, która używa wskaźnika do struktury \textsf{studenCollection} oraz numeru studenta, a następnie usuwa z kolekcji rejestr z tym właśnie numerem.

\item \textbf{Implementacja łańcucha znaków przy użyciu listy powiązanej 2 *}\\
Zaimplementuj funkcję \textsf{concatenate} dla implementacji łańcuchów wykorzystującej listę powiązaną. Pamiętaj, że w przypadku wywołania \textsf{concatenate(s1, s2)}, w którym oba parametry są wskaźnikami do pierwszych węzłów odpowiednich list powiązanych, funkcja powinna kopiować każdy z węzłów \textsf{s2} i dołączać go na koniec \textsf{s1}. Oznacza to, że funkcja nie powinna po prostu przypisać polu next ostatniego węzła na liście \textsf{s1} adresu pierwszego węzła listy \textsf{s2}.

\item \textbf{Implementacja łańcucha znaków przy użyciu listy powiązanej 3 *}\\
Dodaj funkcję \textsf{removeChars} do implementacji łańcuchów wykorzystującej listę powiązaną. Funkcja powinna usuwać część znaków z łańcucha, korzystając z parametrów określających pozycję i długość. Upewnij się, że pamięć wykorzystywana przez usunięte węzły zostanie prawidłowo zwolniona.



\end{enumerate}
\vspace{1cm}
\small Pytania, a także rozwiązania zadań, można wysyłać na adres: \textsc{mdabrowski@fuw.edu.pl}.
\end{document}