\documentclass[12pt]{article}

\usepackage{geometry}
\usepackage{amsmath}
\usepackage{amssymb}
\usepackage{enumitem}
\usepackage{fancyhdr}
\usepackage[utf8x]{inputenc}
\usepackage[T1]{fontenc}
\usepackage[polish]{babel}
\usepackage{amsmath}

\renewcommand{\headrulewidth}{1.5pt}
\pagestyle{fancy}

\fancyhead[LE,RO]{Share\LaTeX}
\fancyhead[RE,LO]{Guides and tutorials}
\fancyfoot[CE,CO]{\leftmark}
\fancyfoot[LE,RO]{\thepage}

\lhead{\textsc{Programowanie, grupa 1}}
%\chead{mdabrowski@fuw.edu.pl}
\rhead{\textsc{Ćwiczenia 14, 8.06.2017}}

\begin{document}

\small \textsc{Elementy języka C++:} podsumowanie wiadomości o C++, projekt -- symulacja.

\begin{enumerate}

\item \textbf{Symulacja kolejki w supermarkecie}\\
Bank X chce umieścić swój bankomat w supermarkecie Y. Dyrekcja
supermarketu martwi się jednak, czy kolejki ustawiające się do bankomatu nie będą zaburzały
przepływu klientów w sklepie i tym zmniejszały wielkość obrotów w kasach supermarketu.
Rozważają więc ustalenie limitu osób oczekujących w kolejce do bankomatu. Przedstawiciele
banku chcieliby zaś oszacować, jak długo klient będzie czekał w kolejce. My zaś zostaliśmy
poproszeni o napisanie programu symulującego całą sytuację tak, aby ułatwić zainteresowanym proces decyzyjny.

\vspace{0.1cm}
Program symulacyjny powinien pozwalać użytkownikowi na określenie parametrów:
\begin{itemize}
\item \textsf{maksymalnego rozmiaru kolejki},
\item \textsf{liczby symulowanych godzin},
\item \textsf{średniej liczby klientów na godzinę}.
\end{itemize}
Program powinien realizować symulację w pętli, której każdy cykl
reprezentuje jedną minutę czasu symulowanego. W każdym takim minutowym cyklu program
powinien wykonać następujące operacje:
\begin{enumerate}
\item Sprawdzić, czy pojawił się nowy klient. Jeśli tak, dołączyć go do kolejki, o ile nie jest
jeszcze pełna -- jeśli jest, odesłać klienta do domu.
\item Jeśli w danym cyklu zakończyła się obsługa klienta przy bankomacie, wziąć pierwszego
klienta z kolejki. Sprawdzić, jak długo będzie korzystał z bankomatu i ustawić odpowiednio
licznik reprezentujący czas obsługi.
\item Jeśli w danym cyklu twa wciąż obsługa poprzedniego klienta, zmniejszyć licznik reprezentujący czas obsługi
o jeden.
\item W każdym cyklu śledzić parametry symulacji: liczbę obsłużonych klientów, liczbę klientów
odesłanych z braku miejsca w kolejce, łączny czas oczekiwania w kolejce i łączną liczbę
oczekujących klientów.
\end{enumerate}
Po zakończeniu symulacji program powinien wyświetlać zestawienie zawierające:\\
\textsf{liczbę klientów przyjętych},
\textsf{liczbę klientów obsłużonych},
\textsf{liczbę klientów odesłanych},
\textsf{średnią długość kolejki} oraz
\textsf{średni czas oczekiwania}.

Przedstawiciel zainteresowanego banku poinformował
nas, że z prowadzonej statystyki wynikają następujące parametry obsługi: jedna trzecia klientów
jest się w stanie obsłużyć przy bankomacie w ciągu mniej niż minuty, jedna trzecia -- w ciągu
dwóch minut, a reszcie zajmuje to do trzech minut. Odstępy czasowe pomiędzy pojawianiem się
nowych klientów w kolejce są losowe, ale średnia na godzinę wychodzi stała.

\newpage

\item \textbf{Symulacja kolejki w supermarkecie 2}\\
Przetestuj program symulacyjny dla parametrów \textsf{(maksymalny rozmiar kolejki, liczba symulowanych godzin, średnia liczba klientów na godzinę)} wynoszących odpowiednio: \textsf{(10, 100, 15)}, \textsf{(10, 100, 30)} oraz \textsf{(20, 10, 30)}.

\vspace{0.1cm}
Zauważ, że podwojenie liczby klientów zgłaszających się do kolejki w ciągu godziny nie podwaja wcale średniego czasu oczekiwania -- ile wynosi ten czas? Jeszcze gorzej jest wydłużać kolejkę. W dodatku symulacja nie uwzględnia jeszcze jednego czynnika -- frustracji oczekujących, z których wielu zapewne po prostu zaniechałoby oczekiwania.

\item \textbf{Symulacja kolejki w supermarkecie 3}\\
Uruchom symulację kilkukrotnie dla krótszego czasu oczekiwania, na przykład z parametrami \textsf{(10, 4, 30)}. Otrzymane wyniki ilustrują rozmaite odchylenia od średnich, które dają się obserwować w krótszych okresach czasu mimo stałej liczby nowych klientów na godzinę. 

\item \textbf{Symulacja kolejki w supermarkecie 4}\\
Dyrekcja supermarketu przeprowadziła analizę z której wynika, że klienci nie lubią czekać
w kolejce dłużej niż minutę. Korzystając z napisanego programu symulacyjnego znajdź
taką wartość średniej liczby klientów na godzinę, dla której średni czas oczekiwania w kolejce
wyniesie minutę (dla czasów co najmniej 100 godzin).

\item \textbf{Symulacja kolejki w supermarkecie 5 *}\\
Dyrekcja supermarketu chciałby wiedzieć co stałoby się, gdyby obok bankomatu postawić drugi.
Zmodyfikuj program symulacyjny tak, aby obsługiwał dwie kolejki. Załóż
przy tym, że klient będzie dołączał do kolejki pierwszej, jeśli będzie w niej mniej oczekujących
niż w kolejce drugiej -- w przeciwnym razie będzie wybierać kolejkę drugą. Ponownie
znajdź taką wartość liczby klientów na godzinę, która da średni czas oczekiwania rzędu
minuty.

\vspace{0.1cm}
Zauważ, że jest to problem nieliniowy w tym sensie, że podwojenie liczby bankomatów
wcale nie podwaja liczby klientów obsługiwanych w ciągu godziny przy założeniu
maksymalnie jednominutowego oczekiwania.

\item \textbf{Symulacja kolejki w supermarkecie 6 **}\\
Zmodyfikuj program symulacyjny tak, aby obsługiwał $N$ kolejek, gdzie liczba $N$ jest podawana przez użytkownika na standardowe wejście. Ponownie
znajdź taką wartość liczby klientów na godzinę, która da średni czas oczekiwania rzędu
minuty.

\end{enumerate}
\vspace{1cm}
\small Pytania można wysyłać na adres: \textsc{mdabrowski@fuw.edu.pl}. Powodzenia na egzaminie!
\end{document}