\documentclass[12pt]{article}

\usepackage{geometry}
\usepackage{amsmath}
\usepackage{amssymb}
\usepackage{enumitem}
\usepackage{fancyhdr}
\usepackage[utf8x]{inputenc}
\usepackage[T1]{fontenc}
\usepackage[polish]{babel}
\usepackage{multicol}

\renewcommand{\headrulewidth}{1.5pt}
\pagestyle{fancy}

\fancyhead[LE,RO]{Share\LaTeX}
\fancyhead[RE,LO]{Guides and tutorials}
\fancyfoot[CE,CO]{\leftmark}
\fancyfoot[LE,RO]{\thepage}
\setlength{\parindent}{0pt}

\lhead{\textsc{Programowanie}}
%\chead{}
\rhead{\textsc{Egzamin, 30.06.2017}}

\newcounter{zadanie}
\newcommand{\zadanie}{\stepcounter{zadanie}\paragraph*{Zadanie \arabic{zadanie}. (10 pkt)}}

\begin{document}

\small Kod źródłowy rozwiązania każdego z zadań zapisz z oddzielnym pliku *.cpp o nazwie \textsc{indeks}\_\textsc{nr zadania}. Programy nie kompilujące się otrzymują {\bf zero} punktów. Można korzystać z Internetu, jednak zabronione są wszelkie formy kontaktowania się. Powodzenia!

\zadanie
Dany jest ciąg liczb całkowitych zdefiniowany rekurencyjnie jako:
\[
T_0 = 1, \quad T_1 = 1, \quad T_2 = 2, \quad T_n = T_{n-1}+T_{n-2}+T_{n-3}.
\]

Napisz funkcję \textbf{rekurencyjną} służącą do obliczania kolejnych wyrazów ciągu. Następnie zastosuj ją w~programie, który wypisuje do pliku tekstowego \textsf{wyniki.txt} pierwszych $n$ wyrazów ciągu $T_n$. Liczba naturalna $n$ powinna być podawana jako argument wywołania programu. \textit{Wskazówka}: możesz użyć funkcji \textsf{atoi} z biblioteki \textsf{<cstdlib>}. \textsc{Przykład}:
\begin{verbatim}
Podana liczba naturalna n: 3
Zawartość pliku tekstowego: 1  1  2
\end{verbatim}

\begin{verbatim}
Podana liczba naturalna n: 6
Zawartość pliku tekstowego: 1  1  2  4  7  13
\end{verbatim}

\zadanie
Napisz klasę bazową \textsf{Funkcja} reprezentującą dziedzinę $[x_D, x_U]$ dowolnej funkcji. Klasa powinna posiadać pola publiczne przechowujące wartości $x_D$ i $x_U$ ($-10^3\leq x_D, x_U\leq 10^3$) oraz konstruktor dwuargumentowy umożliwiający ich inicjalizację.

\vspace{0.3cm}
Następnie napisz klasy pochodne \textsf{Liniowa} oraz \textsf{Kwadratowa} dziedziczące publicznie po klasie \textsf{Funkcja}, reprezentujące \textbf{miejsca zerowe} funkcji liniowej ($y = Ax + B$) oraz
kwadratowej ($y = Ax^2 + Bx + C$). Każda z nich powinna posiadać pola przechowujące parametry i miejsca zerowe funkcji oraz konstruktor wywołujący metodę \textsf{void mZerowe()}, która znajduje \textbf{miejsca zerowe} funkcji. W przypadku, gdy miejsce zerowe funkcji nie istnieje lub znajduje się poza dziedziną, przypisz mu wartość $\mathrm{Inf}=10^6-1$.

\vspace{0.3cm}
\textsc{Przykład użycia klas:}
\begin{verbatim}
int main() {
   Liniowa fun1(-1.5, 1, 5, 2.5); // [-1.5, 1], A=5, B=2.5
   cout << fun1.m0 << endl; // m0 = -0.5
   Kwadratowa fun2(-2.5, 5, 2, -1, -28); // [-2.5, 5], A=2, B=-1, C=-28
   cout << fun2.m1 << '\t' << fun2.m2 << endl; // m1 = Inf, m2 = 3.5
}
\end{verbatim}
\newpage
\zadanie

Korzystając z klasy \textsf{Naukowiec} (plik nagłówkowy \textsf{Naukowiec.h} oraz źródłowy \textsf{Naukowiec.cpp}) napisz klasę \textsf{Naukowcy} przechowującą informacje o grupie naukowców. Klasa powinna zapewniać następujące operacje:
\begin{itemize}
\item konstruktor jednoargumentowy inicjalizujący nazwę grupy i liczbę członków równą 0,
\item operator \textsf{+=} dodający do grupy podanego naukowca (na dowolnej pozycji),
\item funkcję \textsf{sredIndeks} zwracającą średnią arytmetyczną indeksów wszystkich naukowców,
\item operator predekrementacji \textsf{--} \textsf{--} i preinkrementacji \textsf{++} znajdujące najgorszego i najlepszego naukowca w danej grupie (\textit{patrz}: operator \textsf{<} w klasie \textsc{Naukowiec}),
\item operator \textsf{<} porównujący grupy naukowców. Lepsza jest grupa z większym średnim indeksem, a jeżeli jest taki sam, to ta zawierająca mniejszą liczbę naukowców.
\item operator \textsf{+} tworzący z dwóch oddzielnych grupy naukowców jedną \textit{supergrupę},
\item operator \textsf{<<} wypisywania danych o grupie naukowców (nazwa grupy i liczebność) do strumienia typu \textsf{ostream}. Następnie powinny zostać wypisane dane wszystkich naukowców (imię, publikacje i cytowania).
\end{itemize}
Kod projektu podziel na pliki nagłówkowe i źródłowe (osobne dla dostarczonej klasy \textsf{Naukowiec} i klasy \textsf{Naukowcy}). Przykłady użycia klas znajdują się w plikach \textsf{testNaukowiec.cpp} oraz \textsf{testNaukowcy.cpp}. \textit{Wskazówka}: możesz użyć pojemnika \textsf{<vector>} z biblioteki STL.

\end{document}
