\documentclass[12pt]{article}

\usepackage{geometry}
\usepackage{amsmath}
\usepackage{amssymb}
\usepackage{enumitem}
\usepackage{fancyhdr}
\usepackage[utf8x]{inputenc}
\usepackage[T1]{fontenc}
\usepackage[polish]{babel}
\usepackage{multicol}

\renewcommand{\headrulewidth}{1.5pt}
\pagestyle{fancy}

\fancyhead[LE,RO]{Share\LaTeX}
\fancyhead[RE,LO]{Guides and tutorials}
\fancyfoot[CE,CO]{\leftmark}
\fancyfoot[LE,RO]{\thepage}
\setlength{\parindent}{0pt}

\lhead{\textsc{Programowanie}}
%\chead{}
\rhead{\textsc{Przygotowanie do Egzaminu}}

\newcounter{zadanie}
\newcommand{\zadanie}{\stepcounter{zadanie}\paragraph*{Zadanie przygotowawcze}}

\newcounter{rozw}
\newcommand{\rozw}{\stepcounter{rozw}\paragraph*{Przykładowe rozwiązanie}}

\begin{document}

\small Jedno z zadań na egzaminie będzie korzystało z rozwiązania poniższego zadania. Zachęcamy do jego samodzielnego rozwiązania. Wzorcowy kod znajduje się na kolejnej stronie. \textsc{Powodzenia!}

\zadanie
Napisz klasę \textsf{Naukowiec} zawierającą nazwisko, liczbę publikacji oraz cytowań naukowca. Klasa powinna zapewniać następujące operacje:
\begin{itemize}
\item konstruktor trójargumentowy inicjalizujący nazwisko, liczbę publikacji oraz cytowań,
\item funkcję składową \textsf{indeks} zwracającą średnią liczbę cytowań na jedną publikację,
\item funkcję składową \textsf{publikuj} zwiększającą liczbę publikacji o zadaną wartość,
\item operator preinkrementacji \textsf{++} zwiększający o 1 liczbę cytowań naukowca,
\item operator \textsf{<} porównujący naukowców. Lepszy jest naukowiec z większą liczbą cytowań, a jeżeli są takie same, to ten dla którego funkcja \textsf{indeks} zwraca większą wartość.
\item operator \textsf{<<} wypisywania danych naukowca do strumienia typu \textsf{ostream} oraz operator \textsf{>>} wczytywania danych naukowca ze strumienia \textsf{istream}.
\end{itemize}
\textsc{Przykład użycia klasy:}
\begin{verbatim}
int main() {
   Naukowiec Bogdan("Bogdan",50,300), Czeslaw;
   cin >> Czeslaw;
   cout << Czeslaw.indeks();
   Bogdan.publikuj(3);
   if(Bogdan < ++Czeslaw)
      cout << Czeslaw << endl;
   else
      cout << Bogdan << endl;
}
\end{verbatim}

\newpage

\rozw

\begin{verbatim}
class Naukowiec {
      string _name;
      int _publikacje, _cytowania;
   public:
      Naukowiec();
      Naukowiec (const string &name, int publikacje, int cytowania);
      double indeks();
      void publikuj (int n);
      Naukowiec &operator ++ ();
      friend bool operator < (Naukowiec &first, Naukowiec &second);
      friend ostream &operator << (ostream &stream, const Naukowiec &naukowiec);
      friend istream &operator >> (istream &stream, Naukowiec &naukowiec);
};

Naukowiec::Naukowiec() {}

Naukowiec::Naukowiec (const string &name, int publikacje, int cytowania):
   _name (name), _publikacje (publikacje), _cytowania (cytowania) {}

double Naukowiec::indeks() {
   return double(_cytowania)/_publikacje; }

void Naukowiec::publikuj (int n) {
   _publikacje += n; }

Naukowiec &Naukowiec::operator ++ () {
   _cytowania++;
   return *this; }
	
bool operator < (Naukowiec &first, Naukowiec &second) {
   return first._cytowania < second._cytowania ||
      (first._cytowania == second._cytowania && first.indeks() < second.indeks() ); }
	
ostream &operator << (ostream &stream, const Naukowiec &naukowiec) {
   return stream << naukowiec._name <<' '<< naukowiec._publikacje <<' '
      << naukowiec._cytowania; }

istream &operator >> (istream &stream, Naukowiec &naukowiec) {
   return stream >> naukowiec._name >> naukowiec._publikacje >> naukowiec._cytowania; }
\end{verbatim}

\end{document}
