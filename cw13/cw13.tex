\documentclass[12pt]{article}

\usepackage{geometry}
\usepackage{amsmath}
\usepackage{amssymb}
\usepackage{enumitem}
\usepackage{fancyhdr}
\usepackage[utf8x]{inputenc}
\usepackage[T1]{fontenc}
\usepackage[polish]{babel}
\usepackage{amsmath}

\renewcommand{\headrulewidth}{1.5pt}
\pagestyle{fancy}

\fancyhead[LE,RO]{Share\LaTeX}
\fancyhead[RE,LO]{Guides and tutorials}
\fancyfoot[CE,CO]{\leftmark}
\fancyfoot[LE,RO]{\thepage}

\lhead{\textsc{Programowanie, grupa 1}}
%\chead{mdabrowski@fuw.edu.pl}
\rhead{\textsc{Ćwiczenia 13, 1.06.2017}}

\begin{document}

\small \textsc{Elementy języka C++:} pliki nagłówkowe i źródłowe, kompilacja i konsolidacja.

\begin{enumerate}

\item \textbf{Prosty kalkulator 3}\\
Napisz program zawierający funkcje \textsf{dodaj}, \textsf{odejmij}, \textsf{pomnoz} i \textsf{podziel}. Każda z nich powinna przyjmować dwie liczby całkowite i zwracać wynik działania. Utwórz prosty kalkulator korzystający z tych funkcji. Deklaracje funkcji umieść w pliku nagłówkowym, a ich kod źródłowy z pliku źródłowym, który będzie niezależny od reszty kodu kalkulatora.

\item \textbf{Lista powiązana}\\
Poniżej znajduje się kod \textsf{original.cpp} związany z listą powiązaną. Podziel kod na plik nagłówkowy oraz pliki źródłowe tak, aby nadawał się do wielokrotnego użycia.
\begin{verbatim}
#include <iostream>
using namespace std;

struct Node {
	Node *p_next;
	int value;};

Node* addNode (Node* p_list, int value) {
	Node *p_new_node = new Node;
	p_new_node->value = value;
	p_new_node->p_next = p_list;
	return p_new_node;}

void printList (const Node* p_list) {
	const Node* p_cur_node = p_list;
	while ( p_cur_node != NULL ) {
		cout << p_cur_node->value << endl;
		p_cur_node = p_cur_node->p_next;}}

int main () {
	Node *p_list = NULL;
	for ( int i = 0; i < 10; ++i ) {
		int value;
		cout << "Enter value for list node: ";
		cin >> value;
		p_list = addNode( p_list, value );}
	printList( p_list );}
\end{verbatim}

\newpage

\item \textbf{Kwestionariusz osobowy 2}\\
Korzystając z klasy \textsf{Name} (ćwiczenia 10) napisz program, który wczytuje ze standardowego wejścia listę imion i nazwisk, a potem wypisuje je na standardowe wyjście najpierw w kolejności alfabetycznej, a następnie w kolejności alfabetycznej inicjałów (w tym celu napisz dodatkowo własną funkcję porównującą, której użyjesz jako argumentu funkcji \textsf{sort}). Kod projektu podziel na plik nagłówkowy i pliki źródłowe. Plik nagłówkowy z deklaracją klasy:
\begin{verbatim}
class Name {
	public:
		Name ();
		Name (const string &given,const string &family);
		string initials () const;
		friend bool operator< (const Name &first,const Name &second);
		friend ostream &operator<< (ostream &stream,const Name &Name);
		friend istream &operator>> (istream &stream,Name &Name);
	private:
		string given,family;};
\end{verbatim}

\item \textbf{Figury geometryczne}\\
Dana jest lista figur geometrycznych, będących obiektami wirtualnej klasy bazowej \textsf{Figure} (ćwiczenia 11), w następującym formacie:
\begin{verbatim}
circle red 5
rectangle blue 2 3
\end{verbatim}

Napisz program, który wczytuje tę listę ze standardowego wejścia, a następnie wypisuje na standardowe wyjście tę samą listę posortowaną w kolejności rosnącego pola powierzchni. Przy każdej figurze powinno być dodatkowo wypisane jej pole powierzchni. Podziel kod na pliki nagłówkowe (osobne dla klasy bazowej \textsf{Figure} oraz każdej z klas pochodnych \textsf{Circle} i \textsf{Rectangle}) i pliki źródłowe (także osobne dla każdej z klas).
\end{enumerate}
\vspace{1cm}
\small Zadań domowych wyjątkowo brak. Pytania można wysyłać na adres: \textsc{mdabrowski@fuw.edu.pl}.
\end{document}