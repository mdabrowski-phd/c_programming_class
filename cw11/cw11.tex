\documentclass[12pt]{article}

\usepackage{geometry}
\usepackage{amsmath}
\usepackage{amssymb}
\usepackage{enumitem}
\usepackage{fancyhdr}
\usepackage[utf8x]{inputenc}
\usepackage[T1]{fontenc}
\usepackage[polish]{babel}
\usepackage{amsmath}

\renewcommand{\headrulewidth}{1.5pt}
\pagestyle{fancy}

\fancyhead[LE,RO]{Share\LaTeX}
\fancyhead[RE,LO]{Guides and tutorials}
\fancyfoot[CE,CO]{\leftmark}
\fancyfoot[LE,RO]{\thepage}

\lhead{\textsc{Programowanie, grupa 1}}
%\chead{mdabrowski@fuw.edu.pl}
\rhead{\textsc{Ćwiczenia 11, 18.05.2017}}

\begin{document}

\small \textsc{Elementy języka C++:} dziedziczenie, klasy abstrakcyjne, funkcje wirtualne.

\begin{enumerate}

\item \textbf{Macierzowa optyka geometryczna}\\
W optyce macierzowej każdy układ elementów optycznych ustawionych wzdłuż jednej osi opisany
jest pewną macierzą \textsf{2x2}. W szczególności pusty odcinek o długości $d$ oraz cienka soczewka o ogniskowej
$f$ reprezentowane są odpowiednio macierzami:
\[
S=
  \begin{bmatrix}
    1 & d \\
    0 & 1
  \end{bmatrix}
  \quad \quad
  L=
  \begin{bmatrix}
    1 & 0 \\
    -1/f & 1
  \end{bmatrix}
\]
Jeżeli na osi optycznej bezpośrednio za elementem opisanym macierzą $M_1$ stoi element opisany macierzą
$M_2$, to układowi temu odpowiada macierz $M = M_2M_1$. Promień światła opisany jest dwuwymiarowym
wektorem
\[
u=
  \begin{bmatrix}
    y \\
    \alpha
  \end{bmatrix}
\]
gdzie $y$ oraz $\alpha$ są odpowiednio odległością promienia od osi optycznej oraz kątem między promieniem a
osią. Promień opisany wektorem $u_0$ po przejściu przez układ opisany macierzą $M$ przechodzi w promień
opisany wektorem $u_1=Mu_0$.

Napisz klasę \textsf{Ray} reprezentującą promień światła. Zaimplementuj w niej:
1) bezargumentowy konstruktor zerujący $y$ oraz $\alpha$;
2) dwuargumentowy konstruktor inicjalizujący $y$ oraz $\alpha$ swoimi argumentami;
3) funkcje składowe \textsf{getDistance} oraz \textsf{getAngle} zwracające odpowiednio $y$ i $\alpha$.

Napisz klasę \textsf{System} opisującą układ elementów optycznych. Zaimplementuj w niej bezargumentowy
konstruktor inicjalizujący macierz układu macierzą jednostkowa.
Jako funkcję zaprzyjaźnioną z klasą \textsf{System} zaimplementuj operator \textsf{*} odpowiadający składaniu dwóch
układów ustawionych bezpośrednio jeden za drugim. Jako funkcję zaprzyjaźnioną z obiema klasami zaimplementuj
operator \textsf{*} odpowiadający działaniu układu na promień.

Z klasy \textsf{System} wywiedź klasę \textsf{Space} reprezentująca pusty odcinek osi optycznej. Zaimplementuj w
niej jednoargumentowy konstruktor inicjalizujący długość odcinka swoim argumentem. Z klasy \textsf{System}
wywiedź też klasę \textsf{Lens} reprezentującą soczewkę. Zaimplementuj w niej jednoargumentowy konstruktor
inicjalizujący ogniskową swoim argumentem.

Wszystkie klasy i funkcje powinny być napisane tak, aby działał następujący program:
\begin{verbatim}
int main(){
Ray ray = Lens(50) * Space(150) * Lens(100) * Ray(0.1, 0);
cout << ray.getAngle() << " " << ray.getDistance() << endl;
return 0;}
\end{verbatim}

\newpage
\rhead{\textsc{Zadania domowe 11}}

\item \textbf{Definicja dowolnego ciągu}\\
Zadeklaruj abstrakcyjną klasę \textsf{Sequence} reprezentującą ciąg matematyczny. Zadeklaruj w niej publiczny operator indeksowania zwracający żądany wyraz ciągu. Z klasy tej wyprowadź klasę \textsf{Arithmetic} reprezentującą ciąg arytmetyczny: $a_n=a_0+n\Delta$. Zdefiniuj dla niej konstruktor umożliwiający nadanie wartości parametrom $a_0$ oraz $\Delta$. Z klasy \textsf{Sequence} wyprowadź również klasę \textsf{Fibonacci} reprezentującą ciąg Fibonacciego. Klasy te powinny być tak napisane, aby można je było wykorzystać w programie:
\begin{verbatim}
int main(){
Sequence arithmetic( 2., 3. );
Sequence fibonacci;
for ( int n = 0; n < 10; ++n ){
cout << arithmetic[ n ] << " ";}
cout << endl;
for ( int n = 0; n < 10 ; ++n ){
cout << fibonacci[ n ] << " ";}
cout << endl;}
\end{verbatim}

\item \textbf{Pochodna dowolnej funkcji}\\
Zadeklaruj obstrakcyjną klasę \textsf{Function} reprezentującą funkcję rzeczywistą zmiennej rzeczywistej. Zadeklaruj w niej wirtualny operator wywołania funkcji, który umożliwia obliczenie wartości dowolnej pochodnej funkcji w dowolnym punkcie. Jeżeli rząd pochodnej nie jest podany, to operator powinien zwracać wartość samej funkcji. Z klasy \textsf{Function} wyprowadź klasy \textsf{Exponential} oraz \textsf{Sinus} reprezentujące odpowiednio funkcje $\exp{(\alpha x)}$ oraz $\sin{(\omega x)}$. Zaimplementuj w nich konstruktory inicjalizujące wartości stałych $\alpha$ i $\omega$ oraz konkretyzacje operatora wywołania funkcji odziedziczonego z klasy \textsf{Function}. Wymienione klasy skonstruuj w taki sposób, aby można je było wykorzystać w programie:
\begin{verbatim}
int main(){
Exponential exponential( 2.);
Sine sine( 3.);
Function& expo = exponential;
Function& sinus = sine;
cout << expo( 1. ) << endl;
cout << sinus( 2. , 1 ) << endl;}
\end{verbatim}

\end{enumerate}
\vspace{1cm}
\small Pytania, a także rozwiązania zadań, można wysyłać na adres: \textsc{mdabrowski@fuw.edu.pl}.
\end{document}