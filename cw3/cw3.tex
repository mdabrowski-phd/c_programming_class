\documentclass[12pt]{article}

\usepackage{geometry}
\usepackage{amsmath}
\usepackage{amssymb}
\usepackage{enumitem}
\usepackage{fancyhdr}
\usepackage[utf8x]{inputenc}
\usepackage[T1]{fontenc}
\usepackage[polish]{babel}

\renewcommand{\headrulewidth}{1.5pt}
\pagestyle{fancy}

\fancyhead[LE,RO]{Share\LaTeX}
\fancyhead[RE,LO]{Guides and tutorials}
\fancyfoot[CE,CO]{\leftmark}
\fancyfoot[LE,RO]{\thepage}

\lhead{\textsc{Programowanie, grupa 1}}
%\chead{mdabrowski@fuw.edu.pl}
\rhead{\textsc{Ćwiczenia 3, 16.03.2017}}

\begin{document}

\small \textsc{Elementy języka C++:} tablice, pojemnik STL <vector>, generator liczb losowych.

\begin{enumerate}

\item \textbf{Orzeł czy reszka?}\\
Napisz program symulujący rzut monetą. Uruchom go wiele razy (tzn. wiele losowań w pętli). Czy uzyskane wyniki wyglądają Twoim zdaniem na losowe?

\item \textbf{Element najmniejszy}\\
Napisz program znajdujący indeks najmniejszego elementu tablicy. Program powinien wczytywać długość tablicy, tworzyć tę tablicę w pamięci, wypełniać losowymi liczbami rzeczywistymi z przedziału od 0 do 1, a następnie wypisywać elementy tablicy wraz z indeksami oraz znaleziony indeks i wartość elementu najmniejszego.

\item \textbf{Sito Eratostenesa}\\
Napisz program znajdujący metodą sita Eratostenesa wszystkie liczby pierwsze mniejsze od danej liczby naturalnej. Program powinien czytać tę liczbę ze standardowego wejścia, a wynik wypisywać na standardowe wyjście.

\item \textbf{Sortowanie bąbelkowe}\\
Napisz program sortujący metodą bąbelkową tablicę wczytaną przez użytkownika. Porównujemy najpierw dwa pierwsze elementy i jeśli są w złej kolejności, to zamieniamy je. Następnie robimy to samo z drugim i trzecim elementem, itd. aż do końca tablicy. Jeżeli w takim pojedynczym przebiegu wszystkie pary były w dobrej kolejności to znaczy, że tablica jest już posortowana. Jeśli natomiast musieliśmy wykonać przynajmniej jedną zamianę, to powtarzamy całą procedurę od początku. 

\item \textbf{Obliczenia statystyczne}\\
Niech dana będzie próba losowa $N$ wartości $x_i$. Średnia i odchylenie standardowe z tej próby wynoszą odpowiednio:
\[
\mu_x=\frac{1}{N}\sum_{i=1}^{N} x_i \quad \quad \sigma_x=\sqrt{\frac{1}{N-1}\sum_{i=1}^{N}(x_i-\mu_x)^2}
\]
Napisz program, który pobiera ze standardowego wejścia dodatnie liczby rzeczywiste aż do momentu, gdy użytkownik wpisze wartość $-1$, a następnie wyznacza średnią i odchylenie standardowe podanych liczb. \textit{Wskazówka}: użyj pojemnika <vector> ze standardowej biblioteki szablonów STL.

\item \textbf{Dominanta sondażowa}\\
Dominantą zbioru danych w statystyce nazywamy taką wartość, która występuje w nim najczęściej. Napisz kod, który przetwarza tablicę danych sondażowych, aby ustalić ich dominantę. Odpowiedź na pytanie sondażowe polegała na podawaniu liczby z zakresu od 1 do 10. W sytuacji gdy istnieje wiele dominant, można wybrać dowolną z nich. Program powinien wczytywać odpowiedzi na pytanie ankietowe do momentu, w którym użytkownik wprowadzi wartość 0.

\newpage
\rhead{\textsc{Zadania domowe 3}}

\item \textbf{Kostka sześcienna}\\
Napisz program symulujący serię rzutów kostką sześcienną. Niech program prosi o~podanie liczby rzutów, a~następnie stworzy histogram rezultatów losowania. Zadbaj o~niedeterministyczne zachowanie programu.

\item \textbf{Wicelider}\\
Napisz program znajdujący położenie drugiego co do wielkości elementu tablicy. Program powinien wypełniać tablicę o zadanej długości losowymi liczbami rzeczywistymi z przedziału od 0 do 1, a następnie wypisywać elementy tablicy wraz z indeksami oraz znaleziony indeks i wartość elementu drugiego co do wartości.

\item \textbf{Losowanie Lotto *}\\
Napisz program losujący 6 parami różnych liczb naturalnych z przedziału od 1 do 49 włącznie i wypisujący je w kolejności rosnącej. W programie nie używaj sortowania. \textit{Wskazówka}: użyj pojemnika <set> ze standardowej biblioteki szablonów STL.

\item \textbf{Sortowanie -- do wyboru, do koloru}\\
Sortowanie bąbelkowe jest najgorszym znanym algorytmem sortowania, jednakże prostym w implementacji. Istnieją również inne (lepsze) algorytmy, wśród nich:  sortowanie przez wstawianie i sortowanie przez wybór. Napisz program sortujący tablicę liczb losowych jedną z powyższych metod, jak dla sortowania bąbelkowego.

\item \textbf{Szyfr podstawieniowy}\\
Napisz program używający szyfru podstawieniowego, w którym wszystkie wiadomości składają się z wielkich liter i znaków interpunkcyjnych. Pierwotna wiadomość jest zwana tekstem jawnym, zaś szyfrogram tworzy się poprzez podmianę każdej z liter na inną. Utwórz w programie tablicę typu \textit{const} składającą się z 26 elementów \textit{char} służących do szyfrowania. Program powinien odczytywać tekst jawny i wyprowadzać odpowiadający mu szyfrogram.

\item \textbf{Szyfr podstawieniowy 2 *}\\
Zmodyfikuj powyższy program w taki sposób, by konwertował szyfrogram z powrotem na tekst jawny w celu zweryfikowania kodowania i dekodowania.

\item \textbf{Szyfr podstawieniowy 3 **}\\
Aby jeszcze bardziej utrudnić problem szyfru podstawieniowego, zmodyfikuj program w taki sposób, by zamiast wykorzystywać wbudowaną tablicy wartości \textit{const}, generował w sposób losowy wzorzec szyfrowania. W praktyce oznacza to umieszczenie losowych znaków w każdym elemencie tablicy. Pamiętaj, że dana litera nie może być substytutem samej siebie. Nie możesz także użyć tej samej litery dwukrotnie.

\end{enumerate}
\vspace{1cm}
\small Pytania, a także rozwiązania zadań, można wysyłać na adres: \textsc{mdabrowski@fuw.edu.pl}.
\end{document}